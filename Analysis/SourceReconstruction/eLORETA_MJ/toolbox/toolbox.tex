\documentclass{article}
\usepackage{epsf}
\setlength{\oddsidemargin}{-.5cm}
\setlength{\evensidemargin}{-.5cm}
\setlength{\topmargin}{2cm}
%\textwidth=7.825truein
\textwidth=18cm
%\textheight= 10.425truein
\textheight=21cm

\begin{document}
\title{EEG analysis toolbox}
\author{Guido Nolte}
\maketitle
\section{Introduction}
This toolbox combines programs MATLAB programs for source analysis of EEG data. The user needs to know only very few of these programs which are explained in this documentation. 

It is assumed that the user has some functional data in form of spatial patterns. This could 
be an averaged potential, a cross-spectrum, or a set of patterns as given e.g. by an ICA analysis, etc. Furthermore, the user needs to know the names of the EEG-channels these patterns refer to. 

The programs make forward and inverse calculations and visualize the results on MRI data or 
surface plots of various brain tissues. Anatomical information is provided by the toolbox using some form 
of a 'standard head'. 

The toolbox contains a demo-program (\verb demo.m  ). Just run this demo to get an impression 
of the functionalities.  
\section{Overview}
\subsection{List of programs}
The programs explained here fall into 3 groups which are  strictly separated. 

The first 'group' is a single program, called 
\verb prepare_sourceanalysis .  The input is the 
list of electrode names, and the output is a complicated structure containing all anatomical 
information one can possibly need. 

The second group contains programs to make any kind of calculation. Right now, there are 
4 programs for 1. forward calculations (realistic or 3 spheres), 2. dipole fits of a potential, 3. dipole fits of cross-spectrum, and 4. the RAP-MUSIC scan algorithm.

The third group contains programs for visualization. Right now, there are 4 programs for 1. contour plotting of a potential, 2. contour plots for cross-spectra (and such), 3. surface plots (plus sources),
and 4. MRI plots (plus sources). 

Here's a list of programs:

\begin{table*}[h]
	\centering
		\begin{tabular}{l|l}
  		Name & Purpose  \\ \hline \hline  
	  	\verb prepare_sourceanalysis   & Collect anatomical infos \\ \hline
	  	\verb forward_general & makes forward calculations \\
	  	\verb dipole_fit_genera & fits n dipoles to a single pattern \\
	  	\verb dipole_fit_cs  & fits n dipoles to a cross-spectrum \\ 
	  	\verb rapmusic  & makes a RAP-MUSIC search \\ \hline
	  	\verb showfield           & make contour plots of potentials\\
		  	\verb showcs           & make 'head-in-head' plots of cross-spectra and such \\
		  	\verb showsurface       & make surface plots of tissues (plus sources ) \\
		  	\verb showmri      & show mri-slices (plus sources) \\ \hline \hline
		  	
		\end{tabular}
\end{table*}
 
\subsection{General structure of programs}
The programs typically have few arguments. Apart from two exceptions, the necessary arguments are always simple numbers, vectors,  matrices, tensors or strings. All optional modifications (e.g. the specification of shown MRI-slices) are contained in a single structure, where the fields control the options. The two exceptions are \verb forward_general   and \verb showmri  which need complicated but fixed (i.e. pureley anatomical) structures contained in the output of \\  \verb prepare_sourceanalysis  .    
\section{Programs} 
\subsection{Preparation}
To prepare the source analysis you need  a list of the electrode-names you use, e.g. named \verb clab  with elements \verb clab{1}='Oz'  ,\verb clab{2}='C3'  , etc.. Then call\\
\verb sa=prepare_sourceanalysis(clab)  ;

The structure \verb sa  contains all relevant anatomical information for the selected set of electrodes with respect to a fixed head. This head is taken as an example data set from the program CURRY. Right now, there are two models available: this one is called 'curry1', and the second one, called 'montreal' 
contains the Montreal standard head obtained from an average of many subjects. The curry-head looks nicer but the Montreal scientifically more sound. To choose the Montreal head enter.
\verb sa=prepare_sourceanalysis(clab,'montreal')  ;
 
The meaning of the fields of the output \verb sa  is mostly obvious. Perhaps not so obvious is:
\begin{enumerate}
\item 
\verb sa.fp  is a structure containing all information needed to make a forward calculation in the  realistic volume conductor.
\item \verb sa.fp_sphere  same as \verb sa.fp  for a 3-shell spherical volume conductor.
\item \verb sa.locs_2D  contains information about electrode location in a 2D projection in the second and third column, the fourth and fifth column contain slightly shifted coordinates to avoid overlapping sphere in the 'head-in-head-plots'. 
\item \verb sa.mri  is a structure containg all information to show MRI-slices (eventually plus sources).
\item \verb sa.myinds  appears only if you provided names of electrodes which are not known to the toolbox (it knows 118 names). The \verb sa.myinds  is the list of channel indices which are known and which will be used. From your patterns you must select these channels by hand, e.g. by \\  \verb potential=potential(sa.myinds)  . 
\end{enumerate}



 
\subsection{Forward and  Inverse calculations}
\subsubsection{Forward calculation }
Program: \verb forward_general  \\
This programs makes all forward calculation essentially by calling the right programs. Usage:\\ 
\verb v=forward_general(dips,fp)  \\  

Input:\begin{itemize}
\item \verb dips  Nx6 matrix. The first 3 numbers of each row denote the dipole location (in cm) and the next three numbers the dipole amplitude.
\item \verb fp  structure containing all (anatomical) information necessary to make the forward calculation, including the name of the program which does the calculation. \verb fp  is contained in the output (\verb sa  ) of \verb prepare_sourceanalysis  as \verb sa.fp  (for realistic volume conductor) or \verb sa.fp_sphere  (for a spherical volume conductor). Both models assume 3 shells with conductivity ratios 1:.02:1. 
\end{itemize}

Output:\begin{itemize}
\item \verb v  is an MxN matrix where M is the number of channels, and the i.th column is the potential of the i.th dipole.
\end{itemize} 


E.g., \verb v=forward_general(randn(1,6),sa.fp);  calculates the potential for a random dipole in a realistic volume conductor.  

{\bf Reference:} In the pre-defined head models, the reference is taken as a channel on the nose, which itself is not included in the list of channels. When you want to use a different reference already in the forward model, you must change \verb fp  . Let \verb A  be an NxN matrix which performs this referencing (N is the number of channels), then set \verb fp.lintrafo=A (or  \verb sa.fp.lintrafo=A  if you keep \verb fp  as part of the bigger structure \verb sa  ). This change of \verb fp  is necessary if you use the forward model for dipole-fitting in a re-referenced system. If you re-reference, be sure that also the patterns are in the new reference. 


\subsection{Fitting  a dipole to a single pattern}
Program: \verb dipole_fit_field  \\ 
This program fits N dipoles to a given pattern. Usage:\\  
\verb [dips,res_out,k,field_theo]=dipole_fit_field(v,fp,ndip);  \\

Input: \begin{itemize}
\item \verb v  is an Mx1 vector denoting amplitudes in M channels.\\ 
\item \verb fp  is the structure for forward caculations.\\ 
\item \verb ndip  is the number of dipoles\\
\end{itemize}
Output:\begin{itemize}
\item \verb dips is an ndipx6 matrix, where each row denotes dipole location and amplitude\\
\item \verb res_out  is the relative error\\
\item \verb k  is the number of  iterations needed.\\   
\item \verb field_theo  is an Mx1 vector, denoting the theoretical potential for the estimated dipole(s) in M channels. 
\end{itemize}

The fit is done using the Levenberg-Marquardt algorithm. As initial condition ndip random dipoles are chosen. Since the optimization problem is not convex, it is recommended to repeat the fit several times and take the one with the smallest error.

The program is a simple least-squares optimization. To use a weighted least squares with a weight matrix 
\verb W  , this can be incorporated into the forward model and into the data, like \verb v->W*v and \verb fp.lintrafo->W*fp.lintrafo . 

\subsection{Fitting  a dipole to a cross-spectrum}   
Program: \verb dipole_fit_cs  \\ 
This program fits N dipoles to a given cross-spectrum (or covariance matrix). Usage:\\  
\verb [dips,c_source,res_out,c_theo]=dipole_fit_cs(cs,fp,ndip,xtype);  \\

Input: 
\begin{itemize}
\item \verb cs  is an MxM (in general complex) matrix vector denoting cross-spectral amplitued in M channels.\\ 
\item \verb fp  is the structure for forward caculations.\\ 
\item  \verb ndip  is the number of dipoles\\
\item \verb xtype  is a string with possible values 'real', 'imag', and 'comp', depending what part of cs should be used.
\end{itemize} 
Output:
\begin{itemize}
\item \verb dips  is an Nx6 matrix, where each row denotes dipole location and amplitude\\
\item \verb c_source   estimaed cross-spectrum of source\\
\item \verb res_out   is the relative error\\ 
\item \verb c_theo  cross-spectrum of the model 
\end{itemize}

The fit is done using the Levenberg-Marquardt algorithm. As initial condition ndip random dipoles are chosen. Since the optimization problem is not convex, it is recommended to repeat the fit several times and take the one with the smallest error.

The program is a simple least-squares optimization. To use a weighted least squares with a weight matrix 
\verb W  , this can be incorporated into the forward model and into the data, like \verb cs->W*cs*W' and \verb fp.lintrafo->W*fp.lintrafo . (In this form, make sure that you do not apply the weight matrix several times to fp.lintrafo.) 

\subsection{RAP Music}   
Program: \verb dipole_fit_cs  \\ 
This program makes a RAP-MUSIC scan. Usage:\\  
\verb [s,vmax,imax,dip_mom,dip_loc]=rapmusic(vv,V,n,grid)  \\ 

 
Input: 
\begin{itemize}
\item \verb vv  is an MxK matrix where each column denotes a spatial pattern,  
\item \verb V  MxLx3 tensor V(:,i,j) denotes the potential of a unit dipole at i.th grid location in j.th direction. \verb V is given in \verb sa.V_fine  (for 5mm grid) and \verb sa.V_coarse   for 1cm grid. \\
 \item \verb n  number of dipoles \\
\item \verb grid   this is optional. grid is an Lx3 matrix denoting the locations for L grid-points. \verb grid    is given in \verb sa.grid_fine  and \verb sa.grid_coarse  . (for fone and coarse grid, respectively)
\end{itemize}
Output:
\begin{itemize}
\item \verb s   Lxns matrix where the i.th column corresponds to the Musing scan over L points projecting out the field of the proceeding i-1 dipoles. \\
\item \verb max  nx1 vector where imax(i) denotes the index of grid point for the i.th dipole. 
\item \verb vmax  Mxn matrix where the i.th column is the potential of the i.th dipole. 
\item \verb dip_mom   nx3 matrix where the i.th row is the moment of the i.th dipole. The i.th dipole is found as that one which maximises s(:,i). Optimization of dipole direction is included in s(:,i). 
\item \verb dip_loc   nx3 matrix denoting the locations of the dipoles. It can only be calculated if a grid was given. Otherwise an empty matrix is returned. 
\end{itemize}

\section{Visualization}
\subsection{Contour maps of potentials} 
Program: \verb showfield \\ 
This makes makes 2D contour-plots of potentials and power-distributions. Usage:\\
\verb showfield(v,locs_2D[,para])  \\


Input:
\begin{itemize}
\item \verb v  Mx1 vector, pattern to be shown for M channels. 
\item \verb locs_D  Mx5 matrix containing 2D channel locations (in the 2nd and 3rd column). \verb locs_2D is provided in \verb sa.locs_2D  .
\item \verb para  optional parameter to specify display options w.r.t. resolution and colorbar. Type \verb help    \verb showfield  for detials.  
\end{itemize}

 
\begin{figure}[tbh]
\begin{center}
%\epsfxsize=\columnwidth
\epsfxsize=12cm
\epsffile{contour1.eps}
\caption{Contour plot of a potential (in default resolution).}
\end{center}
\end{figure}


\subsection{Contour maps of cross-spectra}  

 Program: \verb showcs  \\ 
This makes makes 2D contour-plots of cross-spectra (and such). Usage:\\
\verb showcs(cs,locs_2D[,para])  \\


Input:
\begin{itemize}
\item \verb cs   MxM real valued matrix, which can be e.g. the imaginary part of cross-spectrum, M channels. 
\item \verb locs_D  Mx5 matrix containing 2D channel locations (in the 2nd and 3rd column). \verb locs_2D is provided in \verb sa.locs_2D  .
\item \verb para  optional parameter to specify display options w.r.t. resolution, colorbar, and position and size of circles. Type \verb help    \verb showcs  for details.  
\end{itemize} 

\begin{figure}[tbh]
\begin{center}
%\epsfxsize=\columnwidth
\epsfxsize=12cm
\epsffile{headinhead1.eps}
\caption{Contour plot of the imaginary part of cross-spectrum at 10Hz. Each small circle shows one row of cs as a contour plot.}
\end{center}
\end{figure}

\subsection{Surface plots of cortex, head, etc. }  

 Program: \verb showsurface  \\ 
Makes 3D surface plots plus various kinds of 'sources'. Usage:\\
\verb showsurface(mysurface[,para,source1,source2,...])  \\


Input:
\begin{itemize}
\item \verb mysurface  structure with two fields: mysurface.vc is a list (Nx3 for N surface-nodes) with locations of nodes in cm, and mysurface.tri (Kx3 for K triangles) is a list of location-indices combining to triangles. Surface are provided as \verb sa.cortex  , \verb sa.head  ,and \verb sa.vc{i}  for i=1:3 denoting the shells of the volume conductor. 
\item \verb para  optional structure to define viewing options (like viewpoint, dipole-length, etc). Type help showsurface for details. Set \verb para=[]  if you want to show sources, but use all defaults as viewing options. 
\item \verb source1,...  define dipoles as a Kx6 matrix, points as Kx3 matrix, or just values as a Kx1 vector. For Kx6 or Kx6 the first 3 columns are interpreted as locations in cm. For Kx1, the values are interpreted as values on the surface nodes, and K must then be equal N, the number of nodes. An arbitary number of different kinds of sources can added. But there can be only one Kx1 source. Dipoles or points in one source all have the same color, but are different for different sources. For Kx1 the values are  shown as color on the surface.   
\end{itemize} 

\begin{figure}[tbh]
\begin{center}
%\epsfxsize=\columnwidth
\epsfxsize=12cm
\epsffile{surface1.eps}
\caption{Channel locations plotted on head. The call 
%was \verb showsurface(sa.head,[],sa.locs_3D(:,1:3))  .
}
\end{center}
\end{figure}

\begin{figure}[tbh]
\begin{center}
%\epsfxsize=\columnwidth
\epsfxsize=12cm
\epsffile{surface2.eps}
\caption{Two simulated dipoles, plus MUSIC-scan on cortex obtained from potential patterns with noise added.}
\end{center}
\end{figure}

\subsection{MRI plots (plus sources)}  
Program: \verb showmri  \\ 
Shows MRI-slices  plus various kinds of 'sources'. Usage:\\
\verb showmri(mri[,para,source1,source2,...])  \\


Input:
\begin{itemize}
\item \verb mri  structure containing all information about the MRI rawdata, scales, and coordinate transformation to go from head system to MRI system. This structure is provided as \verb sa.mri .
\item \verb  para  optional structure to define viewing options (like slice-thickness, number of slices, orientation (axial, sagittal (default), and coronal), dipole-length, etc). Type 'help showmri' for details. Set \verb para=[]  if you want to show sources, but use all defaults as viewing options. 
\item \verb source1,...  define dipoles as an Kx6 matrix, valued points (shown as colored squares) as a Kx4 matrix and points as a Kx3 matrix. x, The first 3 columns are interpreted as locations in cm. An arbitary number of different kinds of sources can added. But there can be only one Kx1 source. Dipoles or points in one source have all the same color, but are different for different sources. For Kx1 the values are  shown as color on the surface.   
\end{itemize} 

\begin{figure}[tbh]
\begin{center}
%\epsfxsize=\columnwidth
\epsfxsize=12cm
\epsffile{mri1.eps}
\caption{Points on innermost shell of the volume conductor plotted on MRI-slices. The call was 
 showmri(sa.mri,[],sa.vc$\{$1$\}$.vc(:,1:3)); .}
\end{center}
\end{figure}

\begin{figure}[tbh]
\begin{center}
%\epsfxsize=\columnwidth
\epsfxsize=12cm
\epsffile{mri2.eps}
\caption{Two dipoles shown in selected slices in all directions. The call was
 para.orientation='all'; para.mricenter=dips(1,1:3); showmri(sa.mri,para,dips);}
\end{center}
\end{figure}


\begin{figure}[tbh]
\begin{center}
%\epsfxsize=\columnwidth
\epsfxsize=12cm
\epsffile{mri3.eps}
\caption{MUSIC scan of potentials of two dipoles plus true dipoles on MRI slices. 
The call was showmri(sa.mri,[],musicres,dips), 
where musicres is an Lx4 matrix. First 3 columns are locations and 4th column is equal to $1./(1-s.^2)$  where s is the output of RAPMUSIC for 1 dipole. Only values above 10 are shown. }
\end{center}
\end{figure}

\end{document}
